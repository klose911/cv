\documentclass[CJKutf8,compress,hyperref]{beamer}
\setlength{\parindent}{0pt}
\setlength{\parskip}{1ex plus 0.5ex minus 0.3ex}

\usepackage{CJKutf8}
\usepackage[T1]{fontenc}
\usepackage{color}
\usepackage{beamerthemesplit} 
\usepackage{graphicx} 
\usepackage{amsmath}           
\usepackage{verbatim}                              
\usepackage{listings} 
\usepackage{relsize}
\usepackage{hyperref}
\usetheme{Darmstadt}                              


\setcounter{tocdepth}{3}                          
\setcounter{secnumdepth}{3}                      

\renewcommand{\today}{\number\year 年 \number\month 月 \number\day 日}

\mode<article> % only for the article version      
{                                                                          
  \usepackage{fullpage}                                          
  \usepackage{hyperref}                                         
}

\mode<presentation> { \setbeamertemplate{background canvas}
  [vertical shading][bottom=blue!8,top=blue!15]     
  \usetheme{Darmstadt}
  \usefonttheme[onlysmall]{structurebold}
}

\lstset{basicstyle=\ttfamily\tiny, keywordstyle=\color{blue!70}, commentstyle=\color{red!50!green!50!blue!50}, frame=shadowbox, rulesepcolor=\color{red!20!green!20!blue!20}, xleftmargin=2em,xrightmargin=2em, aboveskip=1em}

\begin{document}
\begin{CJK}{UTF8}{song}
\bibliographystyle{unsrt} 
\title{\CJKfamily{hei} $\lambda$演算}
\author{\CJKfamily{hei} 吴善良}
%% \institute{\CJKfamily{hei} 易保网络技术有限公司}
\date{\CJKfamily{hei} \today}

\frame{\titlepage}
\tableofcontents
\section{\CJKfamily{hei} $\lambda$演算}

\begin{frame}
  \frametitle{\CJKfamily{hei} $\lambda$演算介绍}
  \begin{itemize}
  \item $\lambda$演算可看做是一个简单的语义清楚的形式语言,用来解释复杂的程序设计语言或者计算模型
  \item $\lambda$演算通常包含两部分
    \begin{itemize} 
    \item{语法}:合法表达式({\color{red}$\lambda$}表达式)的形式系统
    \item{语义}:变换规则的形式系统
    \end{itemize}
  \end{itemize}
\end{frame}

\subsection{ \CJKfamily{hei} $\lambda$表达式}

\begin{frame}
  \frametitle{\CJKfamily{hei} $\lambda$表达式:定义}
  \begin{itemize}
  \item $\lambda$表达式由变量名,抽象
    符号$\lambda$ . ( )组成 
    \begin{eqnarray}
      <\textrm{$\lambda$表达式}> & ::= & <\textrm{变量名}> \\
                                 & & \mid\quad <\textrm{$\lambda$表达式}>\quad<\textrm{$\lambda$表达式}> \\ 
                                 & & \mid\quad \lambda<\textrm{变量名}>.<\textrm{$\lambda$表达式}> \\
                                 & & \mid\quad (<\textrm{$\lambda$表达式}>)
    \end{eqnarray}    
  \item 没有类型,没有常量 
  \item 变量名不仅可以代表变量,还可以代表{\color{blue}函数}
  \end{itemize}
\end{frame}

\begin{frame} 
  \frametitle{\CJKfamily{hei} $\lambda$表达式:解释} 
  \begin{itemize}
  \item{E1 E2}: 函数调用, E1是函数名,E2是实参 
  \item{$\lambda$x.E}: 函数抽象,x是形参,E是函数体
  \end{itemize} 

\end{frame}

\section{\CJKfamily{hei}致谢}
\begin{frame}
  \begin{Huge}
    \begin{center}
      谢谢大家的聆听!
    \end{center}
  \end{Huge}
\end{frame}
%% \bibliography{axis.bib} 
\end{CJK}
\end{document}
