\documentclass[CJKutf8,compress,hyperref]{beamer}
\setlength{\parindent}{0pt}
\setlength{\parskip}{1ex plus 0.5ex minus 0.3ex}

\usepackage{CJKutf8}
\usepackage[T1]{fontenc}
\usepackage{color}
\usepackage{beamerthemesplit} 
\usepackage{graphicx} 
\usepackage{amsmath}           
\usepackage{verbatim}                              
\usepackage{listings} 
\usepackage{relsize}
\usepackage{hyperref}
\usetheme{Darmstadt}                              


\setcounter{tocdepth}{3}                          
\setcounter{secnumdepth}{3}                      

\renewcommand{\today}{\number\year 年 \number\month 月 \number\day 日}

\mode<article> % only for the article version      
{                                                                          
  \usepackage{fullpage}                                          
  \usepackage{hyperref}                                         
}

\mode<presentation> { \setbeamertemplate{background canvas}
  [vertical shading][bottom=blue!8,top=blue!15]     
  \usetheme{Darmstadt}
  \usefonttheme[onlysmall]{structurebold}
}

\lstset{basicstyle=\ttfamily\tiny, keywordstyle=\color{blue!70}, commentstyle=\color{red!50!green!50!blue!50}, frame=shadowbox, rulesepcolor=\color{red!20!green!20!blue!20}, xleftmargin=2em,xrightmargin=2em, aboveskip=1em}

\begin{document}
\begin{CJK}{UTF8}{song}
\bibliographystyle{unsrt} 
\title{\CJKfamily{hei}$\lambda$演算}
\author{\CJKfamily{hei}吴善良}
%% \institute{\CJKfamily{hei} 易保网络技术有限公司}
\date{\CJKfamily{hei} \today}

\frame{\titlepage}
\tableofcontents
\section{\CJKfamily{hei}$\lambda$演算}

\begin{frame}
  \frametitle{\CJKfamily{hei}$\lambda$演算介绍}
  \begin{itemize}
  \item $\lambda$演算可看做是一个简单的语义清楚的形式语言,用来解释复杂的程序设计语言或者计算模型
  \item $\lambda$演算通常包含两部分
    \begin{itemize} 
    \item{语法}:合法表达式({\color{red}$\lambda$}表达式)的形式系统
    \item{语义}:变换规则的形式系统
    \end{itemize}
  \end{itemize}
\end{frame}

\subsection{ \CJKfamily{hei}$\lambda$表达式}

\begin{frame}
  \frametitle{\CJKfamily{hei}$\lambda$表达式:定义}
  \newtheorem{LE}{$\lambda$表达式} 
  \begin{LE}
    $\lambda$表达式由变量名,抽象
    符号$\lambda$ . ( )组成 
    \begin{eqnarray*}
      <\textrm{$\lambda$表达式}> & ::= & <\textrm{变量名}> \\
      <\textrm{$\lambda$表达式}> & ::= & (<\textrm{$\lambda$表达式}>\quad<\textrm{$\lambda$表达式}>) \\ 
      <\textrm{$\lambda$表达式}> & ::= & (\lambda<\textrm{变量名}>.<\textrm{$\lambda$表达式}>)  
    \end{eqnarray*} 
  \end{LE}      
  \begin{itemize}
  \item 没有类型,没有常量 
  \item 变量名不仅可以代表变量,还可以代表{\color{red}函数}
  \end{itemize}
\end{frame}

\begin{frame} 
  \frametitle{\CJKfamily{hei}$\lambda$表达式:优先级规则} 
  \begin{itemize}
  \item{E1 E2}: 函数调用, E1是函数名,E2是实参
    \begin{itemize} 
    \item 施用型表达式是左结合规则
      \begin{displaymath} 
        E_1  E_2 E_3 \dots E_n= (((E_1 E_2) E_3) \dots) E_n
      \end{displaymath}  
    \end{itemize} 
  \item{$\lambda$x.E}: 函数抽象,x是形参,E是函数体 
    \begin{itemize} 
    \item 抽象型表达式是右结合规则
      \begin{displaymath} 
        \lambda x_1 \centerdot  \lambda  x_2 \centerdot  
        \dots \lambda x_n \centerdot E = \lambda x_1
        \centerdot (\lambda x_2 \centerdot
        (\dots (\lambda x_n \centerdot E) \dots ) \,)  
      \end{displaymath}  
    \end{itemize} 
  \item 复杂例子 
    \begin{displaymath} 
      \underline{\lambda x_1 x_2 \dots x_n} \centerdot E = \lambda x_1 \centerdot  \lambda  x_2 \centerdot  
      \dots \lambda x_n \centerdot E
    \end{displaymath} 
    \begin{displaymath} 
      \lambda x_1 \centerdot  \lambda  x_2 \centerdot  
      \dots \lambda x_n \centerdot \underline{E_1  E_2
        E_3 \dots E_n} =  \lambda x_1 \centerdot  \lambda  x_2 \centerdot  
      \dots \lambda x_n \centerdot (E_1 E_2 E_3 \dots E_n) 
    \end{displaymath} 
  \end{itemize} 
\end{frame}

\begin{frame} 
  \frametitle{\CJKfamily{hei}$\lambda$表达式:子表达式}  
  \begin{itemize} 
  \item{\color{red}子表达式}:设E是一个$\lambda$表达式,那E的子表达式可以定义为 
    \begin{itemize} 
    \item $E \equiv x$, x就是E的子表达式 
    \item $E \equiv E1 \; E2$,E1,E2就是E的子表达式 
    \item $E \equiv \lambda x \centerdot E' $,$\lambda x
      \centerdot E'$和$E'$的子表达式都是E的子表达式
    \item $E \equiv (E')$ $E'$和$E'$的子表达式都是E的子表达式
    \end{itemize} 
  \item {\color{red}$SUB(E)$}表示E的所有子表达式
  \end{itemize} 
\end{frame} 

\subsection{ \CJKfamily{hei}自由变量}

\begin{frame} 
  \frametitle{\CJKfamily{hei}变量:作用域}  
  \begin{itemize} 
  \item 变量的作用域 
    \begin{itemize} 
    \item $ \lambda x \centerdot E$中的变量x是被绑定的,他的作用域是E中去掉所有形如 $ \lambda x \centerdot E'$子表达式的表达式部分
    \item $ \lambda x \centerdot E$中的$ \lambda x \centerdot$可以看作变量的$x$的{\color{red}定义点},在E中x的作用域出现的x是变量x的{\color{red}使用点} 
    \end{itemize} 
  \item 例子 
    \begin{itemize} 
    \item $\lambda$表达式:$y (\lambda {\color{cyan}x}
      {\color{magenta}y} \centerdot {\color{magenta}\underline{ 
          {\color{cyan}\underline{{\color{black}{y}}}} {\color{black}(\lambda}
          {\color{blue}x} {\color{black}\centerdot} 
          {\color{blue}\underline{{\color{black}{xy}}}}}}))
      (z (\lambda {\color{green}x} \centerdot
      {\color{green}\underline{{\color{black}{xx}}}})) $ 
    \item {\color{cyan}x}的作用域是y 
    \item {\color{magenta}y}的作用域是$y(\lambda x \centerdot xy)$
    \item {\color{blue}x}的作用域是$xy$ 
    \item {\color{green}x}的作用域是$xx$
    \end{itemize}
  \end{itemize} 
\end{frame}

\begin{frame} 
  \frametitle{\CJKfamily{hei}变量:自由出现}  
  \begin{itemize}
  \item $\lambda$表达式中相同的变量名,可以出现在不同位置,他们的含义可能不同
  \item 自由出现:$\lambda$表达式E中的变量名x的一次出现成为自由出现,如果E中任何一个$\lambda x \centerdot E'$的子表达式不包含该出现 
  \item 约束出现:$\lambda$表达式E中的变量名x的一次出现成为约束出现,如果E中存在一个$\lambda x \centerdot E'$的子表达式包含该出现 
  \item ${\color{red}y}(\lambda x {\color{blue}y} \centerdot
    y (\lambda x \centerdot {\color{green}x}y))
    ({\color{magenta}z} (\lambda x \centerdot {\color{cyan}x}x))$ 
    \begin{itemize} 
    \item {\color{red}y} 自由出现
    \item {\color{blue}y} 约束出现 
    \item {\color{green}x} 自由出现 
    \item {\color{magenta}x} 自由出现 
    \item {\color{cyan}x} 约束出现 
    \end{itemize} 
  \end{itemize}  
\end{frame}

\begin{frame}
  \frametitle{\CJKfamily{hei}自由变量:定义} 
  \newtheorem{fv}{自由变量} 
  \begin{fv}
    如果$\lambda$表达式E中至少包含一个变量x的自由出现,则称x
    为E的{\color{red}自由变量}, $FV(E)$表示$\lambda$表达式E
    的自由变量集合 \\ 
    如果$\lambda$表达式E不包含自由变量,即$FV(E) = \emptyset$,则称E为封闭型表达式\\ 
    如果$\lambda$表达式E包含自由变量,即$FV(E) \neq \emptyset$,则称E为开型表达式
  \end{fv} 
  \begin{itemize}
  \item  
    $E \equiv x , FV(E) = \{ x \}$
  \item       
    $E \equiv E1 \; E2,  FV(E) =  FV(E1) \cup FV(E2) $    
  \item
    $E \equiv \lambda x \centerdot E',  FV(E) =  FV(E')
    -\{ x \}     $
  \item
    $E \equiv (E'), FV(E) =  FV(E') $
  \end{itemize}
\end{frame}

\begin{frame}
  \frametitle{\CJKfamily{hei}自由变量:计算FV} 
  \begin{eqnarray*}
    E & = &y (\lambda xy \centerdot y ( \lambda x \centerdot
            x y)) (z (\lambda x \centerdot xx))  \\ 
    FV(E) & = &FV(y (\lambda xy \centerdot y ( \lambda x \centerdot
                x y))) \cup FV((z (\lambda x \centerdot xx))) \\ 
      & = & FV(y) \cup FV((\lambda xy \centerdot y ( \lambda x \centerdot
            x y))) \cup FV(z) \cup FV((\lambda x \centerdot xx))
    \\
      & = & \{ y \} \cup (FV(y(\lambda x \centerdot  x y))
            - \{ x y\}) \cup \{ z \} \cup (FV(xx) - \{ x
            \}) \\ 
      & = & \{ y \} \cup \underbrace{(\{ y \} \cup FV(\lambda x
            \centerdot  x y) - \{x y\})}_{\emptyset} \cup
            \{z\} \cup \emptyset \\ 
      & = & \{y \; z\}  
  \end{eqnarray*} 
\end{frame}

\subsection{\CJKfamily{hei}  变量替换} 
\begin{frame}
  \frametitle{\CJKfamily{hei}变量替换:定义} 
  \newtheorem{subst}{替换} 
  \begin{subst}
    $E$和$E_0$是$\lambda$表达式,$x$是变量名 \\
    {\color{red} 替换}$E[E_0/x]$表示把E中所有x的自由出现替换成$E_0$
  \end{subst}
  \begin{itemize}
  \item 只有自由出现的变量可以被替换,而且{\color{red}替换不应该把变量的自由出现变成约束出现}
  \item $E[E_0/x]$的计算规则 
    \begin{itemize}
    \item {S1:}  $ E \equiv x,  x[E_0/x]  = E_0 $ \label{S1}
    \item{S2:}   $ E \equiv y, x \neq y,   y[E_0/x]  = y  $
    \item {S3:} $ E \equiv (E'),   (E')[E_0/x]  = E'[E_0/x]  $
    \item {S4:} $ E \equiv E_1E_2,   E_1E_2[E_0/x]  =
      (E_1[E_0/x])(E_2[E_0/x]) $
    \item{S5:} $ E \equiv \lambda x \centerdot E',   
      \lambda x \centerdot    E'[E_0/x]   =  \lambda x \centerdot E' $
    \end{itemize}
  \end{itemize}
\end{frame}

\begin{frame}
  \frametitle{\CJKfamily{hei}变量替换:计算规则2} 
  \begin{itemize}
  \item  $E \equiv \lambda y \centerdot E', x \neq y$ 
    \begin{itemize}
    \item {S6:} $E_0$中没有y的自由出现,直接对$E'$进行替换,
      $ y \not \in FV(E_0),   (\lambda y
      \centerdot E') [E_0/x] = \lambda y \centerdot
      (E'[E_0/x]) $
    \item {S7:} $E'$中没有x的自由出现,则E'没有可替换,
      $ x \not \in FV(E'),   (\lambda y \centerdot E')
      [E_0/x] = \lambda y
      \centerdot E' $
    \item{S8:}  $E_0$中有y的自由出现,$E'$中有x的自由出现,则需要对$E_0$中的y进行改名,改变后的变量名z在$E_0$不存在自由出现,   
      \begin{eqnarray*}
        &y \in FV(E_0) \wedge x \in FV(E'), & \\   
        & (\lambda y  \centerdot  E') [E_0/x] 
          = \lambda z  (E'[z/y]  [E_0/x]),  
                                            & z \not \in FV(E_0), z \neq y   
      \end{eqnarray*} 
    \end{itemize}
  \end{itemize}
\end{frame}

\begin{frame}
  \frametitle{\CJKfamily{hei}变量替换:例子}  
  \begin{itemize}
  \item {简单例子}    
    \begin{eqnarray*}
      x[xy/x] = & xy  & (S1) \\ 
      y[M/x] = & y & (S2) \\ 
      (\lambda x \centerdot  xy)[E/x] = & \lambda x \centerdot xy & (S3;S5) \\  
      (\lambda x \centerdot  xz)[w/y] = & \lambda x \centerdot xz & (S3;S7) 
    \end{eqnarray*} 
  \item {复杂例子} 
    \begin{eqnarray*}
      & ({\color{blue}\underline{{\color{black}(\lambda x \centerdot xy)}}} \;  
        {\color{red}\underline{{\color{black}(\lambda b \centerdot xy)}}}) 
        [\lambda a \centerdot ab/y] & \\ 
      = & {\color{blue}\underline{{\color{black}
          (\lambda x \centerdot xy) [\lambda a \centerdot ab/y]}}} \;  
          {\color{red}\underline{{\color{black}(\lambda b \centerdot xy) 
          [\lambda a \centerdot ab/y]}}} & (S4)  \\  
      = & {\color{blue}\underline{{\color{black}
          (\lambda x \centerdot x(\lambda a \centerdot ab))}}} \;
          {\color{red}\underline{{\color{black}(\lambda b \centerdot xy) 
          [\lambda a \centerdot ab/y]}}} & (S6) \\  
      = & (\lambda x \centerdot x(\lambda a \centerdot ab)) \; 
          {\color{red}\underline{{\color{black}(\lambda z \centerdot xy) [z/b]}}}
          [\lambda a \centerdot ab/y] & (S8) \\  
      = & (\lambda x \centerdot x(\lambda a \centerdot ab)) \; 
          {\color{red}\underline{{\color{black}
          (\lambda z \centerdot xy) [\lambda a \centerdot ab/y]}}} & (S7) \\ 
      = & (\lambda x \centerdot x(\lambda a \centerdot ab)) \; 
          (\lambda z \centerdot x(\lambda a \centerdot ab)) & (S6) 
    \end{eqnarray*}
  \end{itemize}
\end{frame}

\section{\CJKfamily{hei}$\lambda$演算的变换系统}
\begin{frame}
  \frametitle{\CJKfamily{hei}变换系统:简介} 
  \begin{itemize}
  \item 变换系统给出了如何从一个$\lambda$表达式转换成和其等价的另一个$\lambda$表达式
  \item 变换系统定义了$\lambda$演算的语义
  \item 不同的$\lambda$演算系统有不同的变换规则
    \begin{itemize}
    \item $\alpha$变换:绑定的变量名称不重要
    \item $\beta$归约:实际上定义了函数调用 
    \item $\eta$变换:函数的外延等价性
    \end{itemize}
  \end{itemize}
\end{frame}

\subsection{\CJKfamily{hei}$\alpha$变换}
\begin{frame}
    \frametitle{\CJKfamily{hei}$\alpha$变换:定义} 
\end{frame} 

\begin{frame}
    \frametitle{\CJKfamily{hei}$\alpha$变换:例子} 
\end{frame} 

\subsection{\CJKfamily{hei}$\beta$归约}
\begin{frame}
    \frametitle{\CJKfamily{hei}$\beta$归约:定义} 
\end{frame} 

\begin{frame}
    \frametitle{\CJKfamily{hei}$\beta$归约:例子} 
\end{frame} 

\subsection{\CJKfamily{hei}$\eta$变换}
\begin{frame}
    \frametitle{\CJKfamily{hei}$\eta$变换:定义} 
\end{frame} 

\begin{frame}
    \frametitle{\CJKfamily{hei}$\eta$变换:例子} 
\end{frame} 

\subsection{\CJKfamily{hei}$\lambda$演算范式}

\section{\CJKfamily{hei}简单类型的$lambda$演算} 
\section{\CJKfamily{hei}递归实现:Y不动子} 

\section{\CJKfamily{hei}致谢}
\begin{frame}
  \begin{Huge}
    \begin{center}
      谢谢大家的聆听!
    \end{center}
  \end{Huge}
\end{frame}
%% \bibliography{axis.bib} 
\end{CJK}
\end{document}
