\documentclass[CJKutf8,compress,hyperref]{beamer}
\setlength{\parindent}{0pt}
\setlength{\parskip}{1ex plus 0.5ex minus 0.3ex}

\usepackage{CJKutf8}
\usepackage[T1]{fontenc}
\usepackage{color}
\usepackage{beamerthemesplit} 
\usepackage{graphicx} 
\usepackage{amsmath}           
\usepackage{verbatim}                              
\usepackage{listings} 
\usepackage{relsize}
\usepackage{hyperref}
\usetheme{Darmstadt}                              


\setcounter{tocdepth}{3}                          
\setcounter{secnumdepth}{3}                      

\renewcommand{\today}{\number\year 年 \number\month 月 \number\day 日}

\mode<article> % only for the article version      
{                                                                          
  \usepackage{fullpage}                                          
  \usepackage{hyperref}                                         
}

\mode<presentation> { \setbeamertemplate{background canvas}
  [vertical shading][bottom=blue!8,top=blue!15]     
  \usetheme{Darmstadt}
  \usefonttheme[onlysmall]{structurebold}
}

\lstset{basicstyle=\ttfamily\tiny, keywordstyle=\color{blue!70}, commentstyle=\color{red!50!green!50!blue!50}, frame=shadowbox, rulesepcolor=\color{red!20!green!20!blue!20}, xleftmargin=2em,xrightmargin=2em, aboveskip=1em}

\begin{document}
\begin{CJK}{UTF8}{song}
\bibliographystyle{unsrt} 
\title{\CJKfamily{hei}$\lambda$演算}
\author{\CJKfamily{hei}Klose Wu}
% \institute{\CJKfamily{hei} 易保网络技术有限公司}
\date{\CJKfamily{hei} \today}

\frame{\titlepage}
\tableofcontents
\section{\CJKfamily{hei}$\lambda$表达式}

\begin{frame}
  \frametitle{\CJKfamily{hei}$\lambda$演算介绍}
  \begin{itemize}
  \item $\lambda$演算可看做是一个简单的语义清楚的形式语言,用来解释复杂的程序设计语言或者计算模型
  \item $\lambda$演算通常包含两部分
    \begin{itemize} 
    \item{语法}:合法表达式({\color{red}$\lambda$}表达式)的形式系统
    \item{语义}:变换规则的形式系统
    \end{itemize}
  \end{itemize}
\end{frame}

\subsection{ \CJKfamily{hei}$\lambda$表达式}

\begin{frame}
  \frametitle{\CJKfamily{hei}$\lambda$表达式:定义}
  \newtheorem{LE}{$\lambda$表达式} 
  \begin{LE}
    $\lambda$表达式由变量名,抽象符号$\lambda$  $\centerdot$  $($  $)$组成 
    \begin{eqnarray*}
      <\textrm{$\lambda$表达式}> & := & <\textrm{变量名}> \\
      <\textrm{$\lambda$表达式}> & := & (<\textrm{$\lambda$表达式}>\quad<\textrm{$\lambda$表达式}>) \\ 
      <\textrm{$\lambda$表达式}> & := & (\lambda<\textrm{变量名}>.<\textrm{$\lambda$表达式}>)  
    \end{eqnarray*} 
  \end{LE}      
  \begin{itemize}
  \item 没有类型,没有常量 
  \item 变量名不仅可以代表变量,还可以代表{\color{red}函数}
  \end{itemize}
\end{frame}

\begin{frame} 
  \frametitle{\CJKfamily{hei}$\lambda$表达式:优先级规则} 
  \begin{itemize}
  \item{E1 E2}: 函数调用, E1是函数名,E2是实参
    \begin{itemize} 
    \item 施用型表达式是左结合规则
      \begin{displaymath} 
        E_1  E_2 E_3 \dots E_n= (((E_1 E_2) E_3) \dots) E_n
      \end{displaymath}  
    \end{itemize} 
  \item{$\lambda$x.E}: 函数抽象,x是形参,E是函数体 
    \begin{itemize} 
    \item 抽象型表达式是右结合规则
      \begin{displaymath} 
        \lambda x_1 \centerdot  \lambda  x_2 \centerdot  
        \dots \lambda x_n \centerdot E = \lambda x_1
        \centerdot (\lambda x_2 \centerdot
        (\dots (\lambda x_n \centerdot E) \dots ) \,)  
      \end{displaymath}  
    \end{itemize} 
  \item 复杂例子 
    \begin{displaymath} 
      \underline{\lambda x_1 x_2 \dots x_n} \centerdot E = \lambda x_1 \centerdot  \lambda  x_2 \centerdot  
      \dots \lambda x_n \centerdot E
    \end{displaymath} 
    \begin{displaymath} 
      \lambda x_1 \centerdot  \lambda  x_2 \centerdot  
      \dots \lambda x_n \centerdot \underline{E_1  E_2
        E_3 \dots E_n} =  \lambda x_1 \centerdot  \lambda  x_2 \centerdot  
      \dots \lambda x_n \centerdot (E_1 E_2 E_3 \dots E_n) 
    \end{displaymath} 
  \end{itemize} 
\end{frame}

\begin{frame} 
  \frametitle{\CJKfamily{hei}$\lambda$表达式:子表达式}  
  \begin{itemize} 
  \item{\color{red}子表达式}:设E是一个$\lambda$表达式,那E的子表达式可以定义为 
    \begin{itemize} 
    \item $E \equiv x$, x就是E的子表达式 
    \item $E \equiv E1 \; E2$,E1,E2就是E的子表达式 
    \item $E \equiv \lambda x \centerdot E' $,$\lambda x
      \centerdot E'$和$E'$的子表达式都是E的子表达式
    \item $E \equiv (E')$ $E'$和$E'$的子表达式都是E的子表达式
    \end{itemize} 
  \item {\color{red}$SUB(E)$}表示E的所有子表达式
  \end{itemize} 
\end{frame} 

\subsection{ \CJKfamily{hei}自由变量}

\begin{frame} 
  \frametitle{\CJKfamily{hei}变量:作用域}  
  \begin{itemize} 
  \item 变量的作用域 
    \begin{itemize} 
    \item $ \lambda x \centerdot E$中的变量x是被绑定的,他的作用域是E中去掉所有形如 $ \lambda x \centerdot E'$子表达式的表达式部分
    \item $ \lambda x \centerdot E$中的$ \lambda x \centerdot$可以看作变量的$x$的{\color{red}定义点},在E中x的作用域出现的x是变量x的{\color{red}使用点} 
    \end{itemize} 
  \item 例子 
    \begin{itemize} 
    \item $\lambda$表达式:$y (\lambda {\color{cyan}x}
      {\color{magenta}y} \centerdot {\color{magenta}\underline{ 
          {\color{cyan}\underline{{\color{black}{y}}}} {\color{black}(\lambda}
          {\color{blue}x} {\color{black}\centerdot} 
          {\color{blue}\underline{{\color{black}{xy}}}}}}))
      (z (\lambda {\color{green}x} \centerdot
      {\color{green}\underline{{\color{black}{xx}}}})) $ 
    \item {\color{cyan}x}的作用域是y 
    \item {\color{magenta}y}的作用域是$y(\lambda x \centerdot xy)$
    \item {\color{blue}x}的作用域是$xy$ 
    \item {\color{green}x}的作用域是$xx$
    \end{itemize}
  \end{itemize} 
\end{frame}

\begin{frame} 
  \frametitle{\CJKfamily{hei}变量:自由出现}  
  \begin{itemize}
  \item $\lambda$表达式中相同的变量名,可以出现在不同位置,他们的含义可能不同
  \item 自由出现:$\lambda$表达式E中的变量名x的一次出现成为自由出现,如果E中任何一个$\lambda x \centerdot E'$的子表达式不包含该出现 
  \item 约束出现:$\lambda$表达式E中的变量名x的一次出现成为约束出现,如果E中存在一个$\lambda x \centerdot E'$的子表达式包含该出现 
  \item ${\color{red}y}(\lambda x {\color{blue}y} \centerdot
    y (\lambda x \centerdot {\color{green}x}y))
    ({\color{magenta}z} (\lambda x \centerdot {\color{cyan}x}x))$ 
    \begin{itemize} 
    \item {\color{red}y} 自由出现
    \item {\color{blue}y} 约束出现 
    \item {\color{green}x} 约束出现 
    \item {\color{magenta}z} 自由出现 
    \item {\color{cyan}x} 约束出现 
    \end{itemize} 
  \end{itemize}  
\end{frame}

\begin{frame}
  \frametitle{\CJKfamily{hei}自由变量:定义} 
  \newtheorem{fv}{自由变量} 
  \begin{fv}
    如果$\lambda$表达式E中至少包含一个变量x的自由出现,则称x为E的{\color{red}自由变量}, $FV(E)$表示$\lambda$表达式E的自由变量集合 \\ 
    如果$\lambda$表达式E不包含自由变量,即$FV(E) = \emptyset$,则称E为封闭型表达式\\ 
    如果$\lambda$表达式E包含自由变量,即$FV(E) \neq \emptyset$,则称E为开型表达式 \\ 
    FV计算规则: 
    \begin{itemize} 
    \item  
      $E \equiv x , FV(E) = \{ x \}$
    \item       
      $E \equiv E1 \; E2,  FV(E) =  FV(E1) \cup FV(E2) $    
    \item
      $E \equiv \lambda x \centerdot E',  FV(E) =  FV(E')
      -\{ x \}     $
    \item
      $E \equiv (E'), FV(E) =  FV(E') $
    \end{itemize}
  \end{fv} 
\end{frame}

\begin{frame}
  \frametitle{\CJKfamily{hei}自由变量:计算FV} 
  \begin{eqnarray*}
    E & = &y (\lambda xy \centerdot y ( \lambda x \centerdot
            x y)) (z (\lambda x \centerdot xx))  \\ 
    FV(E) & = &FV(y (\lambda xy \centerdot y ( \lambda x \centerdot
                x y))) \cup FV((z (\lambda x \centerdot xx))) \\ 
      & = & FV(y) \cup FV((\lambda xy \centerdot y ( \lambda x \centerdot
            x y))) \cup FV(z) \cup FV((\lambda x \centerdot xx))
    \\
      & = & \{ y \} \cup (FV(y(\lambda x \centerdot  x y))
            - \{ x y\}) \cup \{ z \} \cup (FV(xx) - \{ x
            \}) \\ 
      & = & \{ y \} \cup \underbrace{(\{ y \} \cup FV(\lambda x
            \centerdot  x y) - \{x y\})}_{\emptyset} \cup
            \{z\} \cup \emptyset \\ 
      & = & \{y \; z\}  
  \end{eqnarray*} 
\end{frame}

\subsection{\CJKfamily{hei}  变量替换} 
\begin{frame}
  \frametitle{\CJKfamily{hei}变量替换:定义} 
  \newtheorem{subst}{替换} 
  \begin{subst}
    $E$和$E_0$是$\lambda$表达式,$x$是变量名 \\
    {\color{red} 替换}$E[E_0/x]$表示把E中所有x的自由出现替换成$E_0$
  \end{subst}
  \begin{itemize}
  \item 只有自由出现的变量可以被替换,而且{\color{red}替换不应该把变量的自由出现变成约束出现}
  \item $E[E_0/x]$的计算规则 
    \begin{itemize}
    \item {S1:}  $ E \equiv x,  x[E_0/x]  = E_0 $ \label{S1}
    \item{S2:}   $ E \equiv y, x \neq y,   y[E_0/x]  = y  $
    \item {S3:} $ E \equiv (E'),   (E')[E_0/x]  = E'[E_0/x]  $
    \item {S4:} $ E \equiv E_1E_2,   E_1E_2[E_0/x]  =
      (E_1[E_0/x])(E_2[E_0/x]) $
    \item{S5:} $ E \equiv \lambda x \centerdot E',   
      \lambda x \centerdot    E'[E_0/x]   =  \lambda x \centerdot E' $
    \end{itemize}
  \end{itemize}
\end{frame}

\begin{frame}
  \frametitle{\CJKfamily{hei}变量替换:计算规则} 
  \begin{itemize}
  \item  $E \equiv \lambda y \centerdot E', x \neq y$ 
    \begin{itemize}
    \item {S6:} $E_0$中没有y的自由出现,直接对$E'$进行替换,
      $ y \not \in FV(E_0),   (\lambda y
      \centerdot E') [E_0/x] = \lambda y \centerdot
      (E'[E_0/x]) $
    \item {S7:} $E'$中没有x的自由出现,则E'没有可替换,
      $ x \not \in FV(E'),   (\lambda y \centerdot E')
      [E_0/x] = \lambda y
      \centerdot E' $
    \item{S8:}  $E_0$中有y的自由出现,$E'$中有x的自由出现,则需要对$E_0$中的y进行改名,改变后的变量名z在$E_0$不存在自由出现,   
      \begin{eqnarray*}
        &y \in FV(E_0) \wedge x \in FV(E'), & \\   
        & (\lambda y  \centerdot  E') [E_0/x] 
          = \lambda z  (E'[z/y]  [E_0/x]),  
                                            & z \not \in FV(E_0), z \neq y   
      \end{eqnarray*} 
    \end{itemize}
  \end{itemize}
\end{frame}

\begin{frame}
  \frametitle{\CJKfamily{hei}变量替换:例子}  
  \begin{itemize}
  \item {简单例子}    
    \begin{eqnarray*}
      x[xy/x] = & xy  & (S1) \\ 
      y[M/x] = & y & (S2) \\ 
      (\lambda x \centerdot  xy)[E/x] = & \lambda x \centerdot xy & (S3;S5) \\  
      (\lambda x \centerdot  xz)[w/y] = & \lambda x \centerdot xz & (S3;S7) 
    \end{eqnarray*} 
  \item {复杂例子} 
    \begin{eqnarray*}
      & ({\color{blue}\underline{{\color{black}(\lambda x \centerdot xy)}}} \;  
        {\color{red}\underline{{\color{black}(\lambda b \centerdot xy)}}}) 
        [\lambda a \centerdot ab/y] & \\ 
      = & {\color{blue}\underline{{\color{black}
          (\lambda x \centerdot xy) [\lambda a \centerdot ab/y]}}} \;  
          {\color{red}\underline{{\color{black}(\lambda b \centerdot xy) 
          [\lambda a \centerdot ab/y]}}} & (S4)  \\  
      = & {\color{blue}\underline{{\color{black}
          (\lambda x \centerdot x(\lambda a \centerdot ab))}}} \;
          {\color{red}\underline{{\color{black}(\lambda b \centerdot xy) 
          [\lambda a \centerdot ab/y]}}} & (S6) \\  
      = & (\lambda x \centerdot x(\lambda a \centerdot ab)) \; 
          {\color{red}\underline{{\color{black}(\lambda z \centerdot xy) [z/b]}}}
          [\lambda a \centerdot ab/y] & (S8) \\  
      = & (\lambda x \centerdot x(\lambda a \centerdot ab)) \; 
          {\color{red}\underline{{\color{black}
          (\lambda z \centerdot xy) [\lambda a \centerdot ab/y]}}} & (S7) \\ 
      = & (\lambda x \centerdot x(\lambda a \centerdot ab)) \; 
          (\lambda z \centerdot x(\lambda a \centerdot ab)) & (S6) 
    \end{eqnarray*}
  \end{itemize}
\end{frame}

\section{\CJKfamily{hei}变换系统}
\begin{frame}
  \frametitle{\CJKfamily{hei}变换系统:简介} 
  \begin{itemize}
  \item 变换系统给出了如何从一个$\lambda$表达式转换成和其等价的另一个$\lambda$表达式
  \item 变换系统定义了$\lambda$演算的语义
  \item 不同的$\lambda$演算系统有不同的变换规则
    \begin{itemize}
    \item $\alpha$变换:绑定的变量名称不重要
    \item $\beta$归约:实际上定义了函数调用 
    \item $\eta$变换:函数的外延等价性
    \end{itemize}
  \end{itemize}
\end{frame}

\subsection{\CJKfamily{hei} $\alpha$变换}
\begin{frame}
  \frametitle{\CJKfamily{hei}$\alpha$变换:定义} 
  \newtheorem{alphaTrans}{$\alpha$变换} 
  \begin{alphaTrans}
    设E是$\lambda$表达式,x,y是变量名,如果$y \not \in FV(\lambda x \centerdot E)$,则称下面变换为{\color{red}$\alpha$变换} 
    \begin{displaymath}
      \lambda x \centerdot E \stackrel{\alpha}{\longrightarrow} \lambda y \centerdot(E[y/x]) 
    \end{displaymath}
  \end{alphaTrans} 
  \begin{itemize}
  \item $\alpha$变换只是改变了$\lambda x \centerdot E$的形参名 
  \item{条件}:{\color{red}新的形参不允许是函数体的自由变量},否则会改变函数含义
  \end{itemize}
\end{frame} 

\begin{frame}
  \frametitle{\CJKfamily{hei}$\alpha$变换:例子} 
  \begin{itemize}
  \item 合法的$\alpha$变换 
    \begin{eqnarray*}
      \lambda x \centerdot (zx) & \stackrel{\alpha}{\longrightarrow} & \lambda y \centerdot (zy) \\ 
      \lambda x \centerdot ((\lambda y \centerdot yx) x) & \stackrel{\alpha}{\longrightarrow} & \lambda z  \centerdot ((\lambda y \centerdot yz) z) 
    \end{eqnarray*}
  \item 非法的$\alpha$变换
    \begin{eqnarray*}      
      \lambda x \centerdot (zy) & \stackrel{\alpha}{\longrightarrow} &  \lambda y \centerdot (zy) \\
      \lambda x \centerdot (z (\lambda y \centerdot x))  & \stackrel{\alpha}{\longrightarrow} & \lambda y \centerdot (z (\lambda y \centerdot y)) 
    \end{eqnarray*}
  \end{itemize}
\end{frame} 

\subsection{\CJKfamily{hei} $\beta$变换}
\begin{frame}
  \frametitle{\CJKfamily{hei}$\beta$变换:定义} 
  \newtheorem{betaRed}{$\beta$变换}
  \begin{betaRed}
    设$(\lambda x \centerdot E)$和$E_0$是$\lambda$表达式,则称下面的变换为{\color{red}$\beta$变换}
    \begin{displaymath}
      (\lambda x \centerdot E)E_0 \stackrel{\beta}{\longrightarrow} E[E_0/x] 
    \end{displaymath} 
  \end{betaRed}
  \begin{itemize}
  \item $\beta$变换事实上定义了{\color{red}函数调用}的语义 
  \item $\beta$变换是最重要的一个变换
  \end{itemize}
\end{frame} 

\begin{frame}
  \frametitle{\CJKfamily{hei}$\beta$变换:例子} 
  \begin{itemize}
  \item $(\lambda x \centerdot xy) x \stackrel{\beta}{\longrightarrow} xy$
  \item $(\lambda x \centerdot xx) y \stackrel{\beta}{\longrightarrow} yy$
  \item  
    \begin{eqnarray*}
      {\color{blue}\underline{{\color{black}(\lambda x \centerdot (\lambda y \centerdot (\lambda z \centerdot xyz)))A}}}BC & & \\
                                                                                                                           & \stackrel{\beta}{\longrightarrow} & {\color{blue}\underline{{\color{black}(\lambda y \centerdot (\lambda z \centerdot Ayz))B}}}C \\
                                                                                                                           & \stackrel{\beta}{\longrightarrow} & {\color{blue}\underline{{\color{black}(\lambda z \centerdot ABz)C}}} \\ 
                                                                                                                           & \stackrel{\beta}{\longrightarrow} & ABC 
    \end{eqnarray*} 
  \end{itemize}
\end{frame} 

\subsection{\CJKfamily{hei} $\eta$变换}
\begin{frame}
  \frametitle{\CJKfamily{hei}$\eta$变换:定义} 
  \newtheorem{etaRed}{$\eta$变换} 
  \begin{etaRed}
    设$\lambda x \centerdot Mx$是一个$\lambda$表达式,且$x \not \in FV(M)$, 则称下面的变换是{\color{red}$\eta$变换}
    \begin{displaymath}
      (\lambda x \centerdot Mx) \stackrel{\eta}{\longrightarrow} M 
    \end{displaymath} 
  \end{etaRed}
  \begin{itemize}
  \item 函数的外延等价性:$\forall x, f(x) = h(x) \Longrightarrow f \equiv h$  
  \item $\eta$变换不是$\lambda$演算系统必须的变换 
  \item 例子 
    \begin{itemize}
    \item{合法变换:} $\lambda x \centerdot (\lambda y \centerdot yy) x \stackrel{\eta}{\longrightarrow}(\lambda y \centerdot yy)$ 
    \item{非法变换:}  $\lambda x \centerdot (\lambda y \centerdot y{\color{red}x}) x {\color{red}\stackrel{\eta}{\nrightarrow}} (\lambda y \centerdot yx) $
    \item 与$\beta$变换结合
      \begin{displaymath}
        \forall y, x \not \in FV(M), (\lambda x \centerdot Mx)y \stackrel{\beta}{\longrightarrow} My
      \end{displaymath}
    \end{itemize}
  \end{itemize}
\end{frame} 

\subsection{\CJKfamily{hei} 归约和范式}
\begin{frame}
  \frametitle{\CJKfamily{hei}归约:定义} 
  \newtheorem{reduction}{归约} 
  \begin{reduction} 
    \begin{itemize}
    \item $(\lambda x \centerdot E)E_0$被称为$\beta$基
    \item $(\lambda x \centerdot Mx)$被称为$\eta$基
    \item $\beta$基和$\eta$基被统称为归约基 
    \end{itemize} 
    对表达式中某一归约基实行某种变换被称为{\color{red}归约}
  \end{reduction} 
  \begin{itemize}
  \item 表达式可以同时有多个归约基 
  \item 归约过程不唯一
  \item {\color{red}不同的归约过程得到的结果不一定相同}
  \end{itemize}
\end{frame} 

\begin{frame}
  \frametitle{\CJKfamily{hei}归约:不同的归约过程得到相同的结果} 
  \label{parad}
  \begin{eqnarray*}
    (\lambda y \centerdot y({\color{blue}\underline{(\lambda a \centerdot xa) (\lambda a \centerdot a))}})\;(\lambda b \centerdot b) & \stackrel{\beta}{\longrightarrow} & {\color{red}\underline{(\lambda y \centerdot y{\color{blue}(x (\lambda a \centerdot a)})) \; (\lambda b \centerdot b)}}   \\ 
                                                                                                                                     &  \stackrel{\beta}{\longrightarrow} & {\color{cyan}\underline{{\color{red}(\lambda b \centerdot b)}\;(x (\lambda a \centerdot a))}} \\
                                                                                                                                     & \stackrel{\beta}{\longrightarrow} & {\color{cyan}x(\lambda a \centerdot a)} \\
    {\color{blue}\underline{(\lambda y \centerdot y((\lambda a \centerdot xa) (\lambda a \centerdot a)))\;(\lambda b \centerdot b)}} & \stackrel{\beta}{\longrightarrow} & {\color{red}\underline{{\color{blue}(\lambda b \centerdot b)}((\lambda a \centerdot xa) (\lambda a \centerdot a))}}   \\ 
                                                                                                                                     &  \stackrel{\beta}{\longrightarrow} & {\color{cyan}\underline{{\color{red}(\lambda a \centerdot xa) (\lambda a \centerdot a)}}} \\
                                                                                                                                     & \stackrel{\beta}{\longrightarrow} & {\color{cyan}x(\lambda a \centerdot a)} \\
    (\lambda y \centerdot y({\color{blue}\underline{(\lambda a \centerdot xa)}} (\lambda a \centerdot a)))\;(\lambda b \centerdot b) &   \stackrel{\eta}{\longrightarrow} & {\color{red}\underline{(\lambda y \centerdot y ({\color{blue}x} (\lambda a \centerdot a))) (\lambda b \centerdot b)}}  \\   
                                                                                                                                     &  \stackrel{\beta}{\longrightarrow} & {\color{cyan}\underline{{\color{red}(\lambda b \centerdot b)}\;(x (\lambda a \centerdot a))}} \\
                                                                                                                                     & \stackrel{\beta}{\longrightarrow} & {\color{cyan}x(\lambda a \centerdot a)}
  \end{eqnarray*}
\end{frame}

\begin{frame}
  \frametitle{\CJKfamily{hei}归约:不同的归约过程得到不同的结果} 
  \label{ambig}
  \begin{itemize}
  \item 归约过程1 
    \begin{eqnarray*}
      {\color{red}\underline{(\lambda x \centerdot y) ((\lambda x \centerdot xx) (\lambda x \centerdot xx))}}  &  \stackrel{\beta}{\longrightarrow} & y[((\lambda x \centerdot xx) (\lambda x \centerdot xx))/x] \\
                                                                                                               &   \stackrel{\beta}{\longrightarrow} & y 
    \end{eqnarray*} 
  \item 归约过程2 
    \begin{eqnarray*}
      (\lambda x \centerdot y) ({\color{red}\underline{(\lambda x \centerdot xx) (\lambda x \centerdot xx)}}) &  \stackrel{\beta}{\longrightarrow} & (\lambda x \centerdot y) ({\color{red}\underline{(\lambda x \centerdot xx) (\lambda x \centerdot xx)}})  \\ 
                                                                                                              &  \stackrel{\beta}{\longrightarrow} & (\lambda x \centerdot y) ({\color{red}\underline{(\lambda x \centerdot xx) (\lambda x \centerdot xx)}}) \\  
                                                                                                              & & \cdots
    \end{eqnarray*} 
  \end{itemize}  
\end{frame}

\begin{frame}
  \frametitle{\CJKfamily{hei}范式:定义} 
  \newtheorem{Paradigm}{范式} 
  \begin{Paradigm}
    如果E是一个$\lambda$表达式,且E不包含任何归约基,这样的表达式被称为{\color{red}范式} \\ 
    如果一个表达式经过有限次归约能成为范式,则称该表达式有范式 \\ 
    最左归约:按归约基的$\lambda$符号出现顺序,每次归约最左边的归约基 
    \begin{itemize}
    \item $X \Rightarrow Y$:经过有限次($\alpha$, $\beta$, $\eta$)变换,X归约成Y
    \item $X \Rightarrow^\gamma Y$:经过有限次($\beta$, $\eta$)变换,X归约成Y
    \item $X \Rightarrow^\alpha Y$:经过有限次$\alpha$变换,X归约成Y
    \end{itemize}
  \end{Paradigm} 
\end{frame} 

\begin{frame}
  \frametitle{\CJKfamily{hei}范式:性质} 
  \newtheorem{Parad_uniq}{范式的唯一性} 
  \begin{Parad_uniq}
    如果有范式,则在$\alpha$变换下一定唯一 
  \end{Parad_uniq}
  \newtheorem{Parad_exist}{范式的存在性} 
  \begin{Parad_exist}
    如果有范式,则最左归约法一定能归约出范式 
  \end{Parad_exist}
  \begin{itemize}
  \item 范式是$\lambda$表达式具有相同解释的最简表达形式 
  \item $\lambda$表达式不一定有范式 例子见第\pageref{ambig}页
  \end{itemize}
\end{frame} 

\section{\CJKfamily{hei}简单类型}

\subsection{\CJKfamily{hei}邱奇数}
\begin{frame}
  \frametitle{\CJKfamily{hei}邱奇数:定义}
  \newtheorem{churchNo}{邱奇数(Church Number)} 
  \begin{churchNo}
    \begin{eqnarray*}
      0 & := & \lambda f \centerdot \lambda x \centerdot x \\ 
      1 & := & \lambda f \centerdot \lambda x \centerdot f \; x \\ 
      2 & := & \lambda f \centerdot \lambda x \centerdot f \; (f \;x) \\ 
      3 & := & \lambda f \centerdot \lambda x \centerdot \underbrace{f \; (f \; (f}_3 \;x))x \\ 
        & \dots &  
    \end{eqnarray*}  
  \end{churchNo}
  \begin{itemize}
  \item 邱奇数是一个高阶函数,它的参数是一个单参数的函数f,返回值也是一个单参数的函数
  \item 邱奇数0是一个恒等函数
  \item 邱奇数n是以函数f作为参数并以f的n次复合调用的函数作为返回值的函数
  \end{itemize}
\end{frame} 

\begin{frame}
  \frametitle{\CJKfamily{hei}邱奇数:运算}  
  \begin{itemize}
  \item SUCC:后继函数,假设$n$一个邱奇数,SUCC函数进行$\beta$归约等价于$n + 1$的邱奇数定义
    \begin{equation*}
      SUCC := \lambda n \centerdot \lambda f \centerdot \lambda x \centerdot f \; (n \; f \; x) 
    \end{equation*}
  \item PLUS:加法函数
    \begin{eqnarray*}
      PLUS & := & \lambda m \centerdot \lambda n \centerdot \lambda f \centerdot \lambda x \centerdot (m \; f \;(n \;f \; x)) \\ 
      PLUS & := & \lambda m \centerdot \lambda n \centerdot (m \; SUCC) \; n  
    \end{eqnarray*}
  \item MULT:乘法函数 
    \begin{eqnarray*}
      MULT & := & \lambda m \centerdot \lambda n \centerdot \lambda f \centerdot m \; (n \; f) \\ 
      MULT & := & \lambda m \centerdot \lambda n \centerdot m \; (PLUS \; n \;0)  
    \end{eqnarray*}
  \end{itemize} 
\end{frame}

\subsection{\CJKfamily{hei}逻辑和谓词}
\begin{frame}
  \frametitle{\CJKfamily{hei}逻辑运算:定义} 
  \newtheorem{boolV}{布尔值(Boolean)}
  \begin{boolV}
    \begin{eqnarray*}
      TRUE  & := & \lambda x \centerdot \lambda y \centerdot x \\
      FALSE & := & \lambda x \centerdot \lambda y \centerdot y 
    \end{eqnarray*} 
  \end{boolV}
  \newtheorem{logicO}{逻辑运算(Logic)} 
  \begin{logicO}
    \begin{eqnarray*}
      AND & := & \lambda p \centerdot \lambda q \centerdot (p \; q \; p) \\
      OR & := & \lambda p \centerdot \lambda q \centerdot (p \; p \; q) \\
      NOT & := & \lambda p \centerdot (p \; FALSE \; TRUE) \\
      IFTHENELSE & := & \lambda p \centerdot \lambda a \centerdot \lambda b \centerdot (p \; a \; b)
    \end{eqnarray*}
  \end{logicO}
\end{frame}

\begin{frame}
  \frametitle{逻辑运算:例子} 
  \begin{eqnarray*}
    & {\color{red}AND \; TRUE \; TRUE} & \\ 
    \equiv & (\lambda p \centerdot \lambda q \centerdot p \; q \; p) \; TRUE \; TRUE & \stackrel{\beta}{\longrightarrow} TRUE \; TRUE \; TRUE \\ 
    \equiv & (\lambda x \centerdot \lambda y \centerdot x) \; TRUE \; TRUE & \stackrel{\beta}{\longrightarrow} {\color{red}TRUE} \\ 
    & {\color{red}AND \; TRUE \; FALSE} & \\ 
    \equiv & (\lambda p \centerdot \lambda q \centerdot p \; q \; p) \; TRUE \; FALSE & \stackrel{\beta}{\longrightarrow} TRUE \; FALSE \; TRUE \\ 
    \equiv & (\lambda x \centerdot \lambda y \centerdot x) \; FALSE \; TRUE & \stackrel{\beta}{\longrightarrow} {\color{red}FALSE} \\ 
    & {\color{red}AND \; FALSE \; TRUE} & \\ 
    \equiv & (\lambda p \centerdot \lambda q \centerdot p \; q \; p) \; FALSE \; TRUE & \stackrel{\beta}{\longrightarrow} FALSE \; TRUE \; FALSE \\ 
    \equiv & (\lambda x \centerdot \lambda y \centerdot y) \; TRUE \; FALSE & \stackrel{\beta}{\longrightarrow} {\color{red}FALSE} \\ 
    \dots \dots  \\ 
    & {\color{red}AND \; FALSE \; FALSE} & \stackrel{\beta}{\longrightarrow} {\color{red}FALSE} \\ 
  \end{eqnarray*}
\end{frame}

\begin{frame}
  \frametitle{\CJKfamily{hei}谓词:定义} 
  \newtheorem{predF}{谓词(Predicate)} 
  \begin{predF}
    谓词是返回布尔值的函数 
  \end{predF}
  \begin{itemize}
  \item ALWAYSFALSE:永远返回FALSE 
    \begin{displaymath}
      ALWAYSFALSE := \lambda x \centerdot FALSE 
    \end{displaymath}
  \item ISZERO:当且仅当其参数为邱奇数0时返回TRUE,否则返回FALSE,\emph{\color{red}FALSE等价于邱奇数0的定义}
    \begin{displaymath}
      ISZERO := \lambda n \centerdot n \; ALWAYSFALSE \; TRUE
    \end{displaymath}
  \end{itemize}
\end{frame}

\begin{frame}
  \frametitle{\CJKfamily{hei}谓词:例子} 
  \begin{itemize}
  \item PRED:前驱元函数
    \begin{displaymath}
      PRED := \lambda n \centerdot \; n \; (\lambda g \centerdot \lambda k \centerdot \; ISZERO \; (g \; 1) \; k \; (PLUS \; (g \; k) \; 1)) \; (\lambda v \centerdot \; 0) \; 0
    \end{displaymath} 
    \begin{itemize}
    \item 根据数学归纳法可以证明当邱奇数$n > 0$的情况下$n \; (\lambda g \centerdot \lambda k \centerdot \; ISZERO \; (g \; 1) \; k \; (PLUS \; (g \; k) \; 1)) \; (\lambda v \centerdot \; 0)$就是加$n-1$次邱奇数1的函数
    \end{itemize} 
  \item SUB:减法函数,根据PRED可以定义 
  \item EQ:比较相等函数,根据SUB可以定义
    \begin{eqnarray*}
      LEQ & :=  & \lambda m \centerdot \lambda n \centerdot \; ISZERO \; (SUB\;m\;n) \quad {\color{red}\textrm{less than or equal}} \\ 
      EQ & := & \lambda m \centerdot \lambda n \; AND \; (LEQ \; m \; n) \; (LEQ \; n \; m) 
    \end{eqnarray*}
  \end{itemize}
\end{frame}

\subsection{\CJKfamily{hei}有序对}
\begin{frame}
  \frametitle{\CJKfamily{hei}有序对:定义} 
  \newtheorem{pairs}{有序对(Pairs)}
  \begin{pairs}
    有序对可以用$TRUE$和$FALSE$来定义 
    \begin{eqnarray*}
      CONS & := & \lambda x \centerdot \lambda y \centerdot \lambda f \centerdot f \; x \; y \\ 
      CAR & := & \lambda p \centerdot p \; TRUE \\
      CDR & := & \lambda p \centerdot p \; FALSE \\ 
      NIL & := & \lambda x \centerdot TRUE \\
      NULL? & := & \lambda p \centerdot p(\lambda x \centerdot \lambda y \centerdot FALSE) \\
    \end{eqnarray*} 
  \end{pairs}
  \begin{itemize}
  \item LIST:列表函数,可以被定义成空列表NIL,或者CONS一个表达式和一个列表 
  \item ATOM?:判断变量是否原子类型函数,当某个变量的CDR是NIL的时候,可以认为这个变量是原子类型 
  \end{itemize}
\end{frame}

\section{\CJKfamily{hei}实现递归} 
\subsection{\CJKfamily{hei}Y不动子}
\begin{frame}
  \frametitle{\CJKfamily{hei}实现递归:Y不动子} 
  \begin{itemize}
  \item 递归是用函数自身去定义函数 
  \item $\lambda$演算的函数都是无名函数,表面看不支持递归,但是可以构造特殊的函数来实现递归
    \newtheorem{YComb}{Y不动子} 
    \begin{YComb}
      \begin{eqnarray*}
        Y & := & \lambda g \centerdot (\lambda x \centerdot g(x \;x)) \; (\lambda x \centerdot g(x \;x)) \\ 
        YG & \equiv & (\lambda x \centerdot G(x \;x)) \;  (\lambda x \centerdot G(x \;x)) \\ 
        YG & \equiv & G (\underbrace{(\lambda x \centerdot G(x \;x)) \;  (\lambda x \centerdot G(x \;x))}_{YG}) \\ 
        YG & \equiv & G(YG) 
      \end{eqnarray*} 
      YG被称为G的一个不动点,Y被称为不动子
    \end{YComb}
  \item  {\color{red}任何递归函数都可以被看成是另一个函数的不动点}
  \end{itemize}
\end{frame}

\begin{frame}
  \frametitle{\CJKfamily{hei}实现递归:用Y不动子计算阶乘1} 
  \begin{displaymath}
    G  := \lambda r \centerdot \lambda n \centerdot (IF \; ISZERO(n) \; 1 \; (MULT \; n \; (r \; (SUB \; n \;1)))) 
  \end{displaymath}
  \begin{eqnarray*}
    & (YG) \; 4  & \\
    & =  &  (G \; (YG)) \; 4 \\  
    & =  & (\underbrace{(\lambda r \centerdot \lambda n \centerdot  (IF \, ISZERO(n) \, 1 \, (MULT \, n \, (r \, (SUB \, n \,1)))))}_G \; YG) \; 4 \\
    & =  & (\lambda n \centerdot (IF \; ISZERO(n) \; 1 \; (MULT \; n ((YG) \; (SUB \; n \; 1))))) \;4 \\
    & =  & IF \; ISZERO(4) \; 1 \; {\color{red}\underline{(MULT \; 4 \; ((YG) \; (SUB \;4 \;1)))}} \\
    & =  & MULT \; 4 \; ((YG) \;3)  \\ 
    & \dots & \\ 
    & = & MULT \; 4 \; (MULT \; 3 \; (MULT \; 2 \; (MULT \; 1 \; {\color{red}((YG) 0)})))   
  \end{eqnarray*}
\end{frame}

\begin{frame}
  \frametitle{\CJKfamily{hei}实现递归:用Y不动子计算阶乘2} 
  \begin{eqnarray*} 
    & {\color{red}(YG) \; 0} &    \\
    & =  & (G \; (YG)) \; 0 \\  
    & = & (\underbrace{(\lambda r \centerdot \lambda n \centerdot  (IF \, ISZERO(n) \, 1 (MULT \, n (r (SUB \, n \,1)))))}_G YG) \, 0 \\
    & =  & (\lambda n \centerdot (IF \; ISZERO(n) \; 1 \; (MULT \; n ((YG) \; (SUB \; n \; 1))))) \;0 \\
    & = & IF \; ISZERO(0) \; {\color{red}\underline{1}} \; (MULT \; 0 \; ((YG) \; (SUB \;0 \;1))) \\ 
    & =  & {\color{red}1} \\
    & (YG) \; 4 &  \\ 
    & = &  MULT \; 4 \; (MULT \; 3 (MULT \; 2 \; (MULT \; 1 \; {\color{red}1}))) \\ 
    & =  & {\color{red}24}
  \end{eqnarray*} 
  {\color{red}定义一个合适的函数G(使用一个额外的参数来描述递归的$\lambda$表达式), 对这个函数G进行不动子求值就相当于调用递归}
\end{frame}

\section{\CJKfamily{hei}$\lambda$计算模型}

\begin{frame}
  \frametitle{\CJKfamily{hei}$\lambda$计算模型:例子} 
  \begin{itemize}
  \item $\lambda$演算可以描述复杂计算,计算能力等价于图灵计算模型 
  \item 给定2个$\lambda$表达式,如果两者等价则输出TRUE,反之则输出FALSE。这是第一个被证明{\color{red}没有算法可以解决}的问题
  \item quote,atom,cons,car,cdr,eq,cond是LISP的7个原始操作符 
    \begin{itemize}
    \item quote : 引用函数,它的自变量不被求值, 而是作为整个表达式的值返回
    \item cond: 条件函数,可以由IFTHENELSE定义
    \item atom,cons,car,cdr,eq:已经被定义
    \item 用这7个操作符可以写出最原始版本的eval函数,也就是最简单的{\color{red}解释器}\cite{RootsOfLisp}
    \end{itemize}
  \end{itemize}
\end{frame} 

\subsection{\CJKfamily{hei}$\lambda$计算模型扩充}
\begin{frame}
  \frametitle{\CJKfamily{hei}$\lambda$计算模型:扩充}
  \begin{itemize}
  \item $\lambda$演算实际使用很不方便 
  \item $\lambda$计算模型扩充
    \begin{itemize}
    \item 扩充表达式
      \begin{itemize}
      \item 常数:TRUE,FALSE,整数
      \item 标准函数:ADD,SUB,MULT,CONS,CAR,CDR ...
      \item 条件表达式:COND((P1 E1) (P2 E2)) 
      \item let表达式:(LET((V1 E1)) E) 
      \end{itemize}
    \item 扩充变换系统 
    \item 扩充数据类型:INT,REAL,BOOLEAN
    \end{itemize}
  \end{itemize}
\end{frame}

\begin{frame}
  \frametitle{\CJKfamily{hei}let表达式} 
  \newtheorem{letE}{let表达式} 
  \begin{letE}
    \begin{displaymath}
      (let ((x \quad E_0)) \quad E_1) \equiv (\lambda x \centerdot E_1) E_0  
    \end{displaymath} 
    把$E_1$中的变量x的值绑定为$E_0$
  \end{letE}
  引入let表达式后$\lambda$演算就有了2种变量 
  \begin{itemize}
  \item $\lambda$变量:形参变量 
  \item let变量:过程体内的临时变量
  \end{itemize}
\end{frame} 

\section{\CJKfamily{hei}致谢}
\begin{frame}
  \begin{Huge}
    \begin{center}
      谢谢大家的聆听!
    \end{center}
  \end{Huge}
\end{frame}
\bibliography{lambda} 
\end{CJK}
\end{document}
